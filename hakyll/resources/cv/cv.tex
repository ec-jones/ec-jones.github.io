\documentclass[10pt]{letter}
\usepackage[british]{babel}

\usepackage[margin=2cm]{geometry}
\pagenumbering{gobble} % Suppress page no.

% Fancy underline
\usepackage{ulem}

% A nicer font
\usepackage{tgpagella}
\usepackage[T1]{fontenc}
\renewcommand{\emph}[1]{\textit{#1}}

% Hyperlinks
\usepackage{xcolor}
\usepackage{hyperref}
\definecolor{UniversityRed}{RGB}{171,31,45}
\hypersetup{
  urlcolor = UniversityRed,
  colorlinks = true
}

% Long version
\newif\iflong
\longtrue

\begin{document}

\begin{minipage}[t]{0.7\textwidth}
  \raggedright{}
  {\huge Eddie Jones}
  \\[10pt]

  \href{https://ec-jones.github.io}{https://ec-jones.github.io}\\
  \href{mailto:eddie.jones@bristol.ac.uk}{eddie.jones@bristol.ac.uk} \\
  \href{https://orcid.org/0000-0003-1762-5405}{ORCiD: 0000-0003-1762-5405}
\end{minipage}
\begin{minipage}[t]{0.29\textwidth}
  \vfill

  \raggedleft{}
  6 Rousham Road\\
  Bristol\\
  BS5 6XJ\\
  +447483 222495
\end{minipage}

\vspace{5pt}

I am a lecturer in programming languages and compilers at the University of Bristol.
My research to date has focused on lightweight methods for automatically verifying functional programs, including refinement type systems, decidable logics, and inductive reasoning systems.

Interests: \textbf{program logics}, \textbf{cyclic proofs}, \textbf{under-approximate reasoning}, \textbf{equational reasoning}.

\vspace{10pt}

\uline{{\large PUBLICATIONS}\hfill}

\vspace{5pt}

\textbf{\href{https://dl.acm.org/doi/abs/10.1145/3571262}{Higher-order MSL Horn Clauses} --- POPL} \hfill January 2023

Authors: Jerome Jochems, Eddie Jones, and Steven Ramsay.\\
Publication: Proceedings of the ACM on Programming Languages, Volume 7, Issue POPL \( \bullet \) \href{https://doi.org/10.1145/3571262}{\textsc{doi}: 10.1145/3571262}.

\vspace{10pt}

\textbf{\href{https://dl.acm.org/doi/10.1145/3519939.3523731}{CycleQ} --- PLDI} \hfill June 2022\\
\emph{an efficient basis for cyclic equational reasoning}

Authors: Eddie Jones, C.-H. Luke Ong, and Steven Ramsay.\\
Publication: Proceedings of the 43rd ACM SIGPLAN International Conference on Programming Language Design and Implementation \( \bullet \) \href{https://doi.org/10.1145/3519939.3523731}{\textsc{doi}: 10.1145/3519939.3523731}.

\vspace{10pt}

\textbf{\href{https://dl.acm.org/doi/10.1145/3434336}{Intensional datatype refinement} --- POPL} \hfill January 2021\\
\emph{with application to scalable verification of pattern-match safety}

Authors: Eddie Jones and Steven Ramsay.\\
Publication: Proceedings of the ACM on Programming Languages, Volume 5, Issue POPL \( \bullet \) \href{https://doi.org/10.1145/3434336}{\textsc{doi}: 10.1145/3434336}.

\vspace{10pt}

\uline{{\large APPOINTMENTS}\hfill}

\vspace{5pt}

\textbf{Lecturer in Programming Languages and Compilers --- University of Bristol} \hfill 2023--\phantom{2023}

\textbf{Research Associate on Taint-Analysis for Erlang --- University of Bristol} \hfill 2023--2023

\iflong
  \vspace{-5pt}
  \begin{itemize}
    \setlength\itemsep{0pt}
    \item With funding from Meta, we pursued an extension of prior work on intensional datatype refinement type system to Erlang.
          This project aims to statically approximate the flow of private information through a program in order to ensure compliance with data protection guidelines.
  \end{itemize}
\fi

\textbf{Program-Level Teaching Assistant --- University of Bristol} \hfill 2021--2022

\iflong
  \begin{itemize}
    \item As a program-level teaching assistant, I led tutorials designed to cross module boundaries and give students a more comprehensive understanding of computer science outside the curriculum. % These flexible sessions were particularly enjoyable as they could be catered to the students' needs and more naturally leads to a rapport.

    \item I also contributed content to this series, designing worksheets on bisimulation and the topological aspects of functional programming languages.
  \end{itemize}
\fi

\textbf{Teaching Assistant --- University of Bristol} \hfill 2018--2023

\iflong
  \begin{itemize}
    \item During my undergraduate degree and PhD, I took the opportunity to be a teaching assistant across a number of units including:
          \begin{center}
            \begin{tabular}{p{0.3\textwidth}p{0.5\textwidth}}
              \begin{itemize}
                \setlength\itemsep{0pt}
                \item Functional Programming
                \item Types \& Lambda Calculus
              \end{itemize}

               &
              \begin{itemize}
                \setlength\itemsep{0pt}
                \item Programming Languages and Computation
                \item Advanced Topics in Programming Languages
              \end{itemize}
            \end{tabular}
          \end{center}

          This role involved leading tutorial-like problem classes, helping the students with lab working, as well as marking homework sheets.

    \item For the Functional Programming and Types \& Lambda Calculus units, I have previously taken on the additional responsibility as lead teaching assistant with associated administrative duties and the task of checking the exam solutions.
  \end{itemize}
\fi

\vspace{10pt}

\uline{{\large EDUCATION}\hfill}

\vspace{5pt}

\textbf{PhD Computer Science --- University of Bristol} \hfill 2019--\phantom{2023}

\iflong
  \vspace{-5pt}
  \begin{itemize}
    \setlength\itemsep{0pt}
    \item Numerous contributions to the research group's seminar series.
    \item Oregon Programming Languages Summer School (2021)
    \item Midlands Graduate School in the Foundations of Computing Science (2021)
  \end{itemize}
\fi

\textbf{BSc (Hons) Mathematics and Computer Science --- University of Bristol} \hfill 2016--2019

\iflong
  \vspace{-5pt}
  \begin{itemize}
    \item During my undergraduate degree, I found that fluency in mathematical thinking gave me the analytical skills necessary to shed new light on the practical challenges faced in computer science.
          I averaged a first-class mark of 85\% across a range of modules including:

          \begin{center}
            \begin{tabular}{p{0.3\textwidth}p{30pt}p{0.3\textwidth}}
              \begin{itemize}
                \setlength\itemsep{0pt}
                \item Language Engineering
                \item Theory of Computation
                \item Types \& Lambda Calculus
                \item AI \& Logic Programming
              \end{itemize}

               &

               &
              \begin{itemize}
                \setlength\itemsep{0pt}
                \item Set Theory
                \item Combinatorics
                \item Dynamic Systems
                \item Machine Learning
              \end{itemize}
            \end{tabular}
          \end{center}

    \item Research experience:
          \begin{itemize}
            \setlength\itemsep{0pt}
            \item \emph{The Dynamics of Dialects}. For my undergraduate dissertation, I used a model of natural language acquisition to investigate, through a series of simulations, how the structure of social networks influence the propagation of cultural symbols. It received a first-class mark of 87\%.
            \item \emph{Applied Optimisation Research Internship}. In my second year as an undergraduate student, I was a research intern in the maths department. This project considered the problem of designing an optimal layout for a car park. It involved a mixture of calculus, geometry, and numerical simulation performed in MATLAB.
          \end{itemize}

    \item Awards:
          \begin{itemize}
            \setlength\itemsep{0pt}
            \item Top Mathematics and Computer Science Graduate 2019
            \item Top 10 Second Year Student in Computer Science, awarded by Netcraft
            \item Top 5 First Year Student in Computer Science, awarded by Bank of America Merrill Lynch
          \end{itemize}
  \end{itemize}
\fi

\textbf{A-Levels --- Peter Symonds College} \hfill 2014--2016

\iflong
  \vspace{-5pt}
  \begin{itemize}
    \setlength\itemsep{0pt}
    \item Mathematics A*
    \item Further Mathematics A
    \item Physics A
    \item (AS) Economics A
  \end{itemize}
\fi

% \textbf{GSCEs --- Swanmore College of Technology} \hfill 2009--2014
% \iflong
% \vspace{-5pt}
% \begin{itemize}
%   \item 13 GCSE including Mathematics, Science, English, and French.
% \end{itemize}
% \fi

\vspace{10pt}

\uline{{\large LANGUAGES \& TOOLS}\hfill}

\vspace{5pt}

\begin{tabular}{p{0.4\textwidth}p{0.2\textwidth}p{0.2\textwidth}}
  \textbf{Advanced:}
  \begin{itemize}
    \setlength\itemsep{0pt}
    \item Haskell
    \item Functional Programming
    \item Mathematics
  \end{itemize}

   &
  \textbf{Experience With:}
  \begin{itemize}
    \setlength\itemsep{0pt}
    \item C
    \item Rust
    \item Python
  \end{itemize}

   &
  \begin{itemize}
    \setlength\itemsep{0pt}
    \item Linux
    \item Git
    \item \LaTeX{}
  \end{itemize}
\end{tabular}

% \vspace{10pt}

% \uline{{\large HOBBIES \& INTERESTS}\hfill}
% \vspace{10pt}

% \begin{adjustwidth}{10pt}{0pt}
%   In addition to my passion for computer science, I thoroughly enjoyvarious physical activities.
%   For many years, I practiced Taekwondo up-to a 2\textsuperscript{nd} Dan black belt, but my current obsessions are squash and bouldering.
%   I also love reading and rarely have fewer than two books on the go.
%   Some of my favourite genres include sci-fi, history and philosophy.
% \end{adjustwidth}

\end{document}